%%% iprp: preambulo
%%% 2006
\usepackage[T1]{fontenc}
\usepackage[scaled]{luximono}
\usepackage{amsmath}
\usepackage{babel}
\usepackage{graphicx}
\usepackage{epstopdf}
\DeclareGraphicsRule{.tif}{png}{.png}{`convert #1 `dirname #1`/`basename #1 .tif`.png}
%\usepackage{picins} % problema com a package
\usepackage[applemac]{inputenc}
\usepackage{natbib}
\usepackage{verbatim}
%\usepackage{verbatim} % acrescentada 07-2007
\usepackage[table]{xcolor}
%\usepackage{colortbl}
\usepackage[colorlinks,urlcolor=blue]{hyperref}
\usepackage{fancybox,calc}
\usepackage{alltt}
\usepackage{subfig}
\usepackage[Lenny]{fncychap}
\usepackage{makeidx}
\usepackage{fancyvrb}
\usepackage{listings}


\usepackage[tikz]{bclogo}
\usepackage{framed}



\newcommand\bcyinyang{\centering \huge \Yingyang}



\lstloadlanguages{Python,C,Java,Prolog,Lisp}

%%% Est�o aqui duas vers�es para o c�digo

\lstnewenvironment{codigo}[2][]{
\lstset{backgroundcolor=\color{cinza},language=Python,numbers = left, numberstyle = \tiny, captionpos=b,showstringspaces=false,basicstyle=\ttfamily,tabsize=2,breaklines=true, #1,#2}}{}

\lstset{backgroundcolor=\color{cinza},language=Python,numbers = left, numberstyle = \tiny, captionpos=b,showstringspaces=false,basicstyle=\ttfamily,commentstyle=\color{blue},tabsize=2,breaklines=true}

%\lstset{backgroundcolor=\color{green},language=Java,captionpos=b,showstringspaces=false,basicstyle=\ttfamily,commentstyle=\color{blue},tabsize=2,breaklines=true}

\usepackage{pifont}
\usepackage{lettrine}
\usepackage{soul}
\usepackage{epigraph}
\usepackage{marvosym}
\usepackage{xcolor}
\usepackage{placeins}
%\usepackage{appendix}

\usepackage[stable]{footmisc} % para colocar notas de rodap� nas sec��es

\newenvironment{objectivos}
	{\noindent {\bfseries{\Large{Objectivos}}}
		\begin{dinglist}{"34}}
	{\end{dinglist}}

\newenvironment{figura}[4]
	{	
	\begin{figure}[!htb]
		\centering
		\includegraphics[scale=#1]{#2} 
		\caption{#3}
		\label{#4}
	}
	{\end{figure}}
	

\newcommand{\python}{\texttt{Python }}



\renewcommand\lstlistingname{Listagem}
\renewcommand\lstlistlistingname{Listagens de C�digo}

\bibliographystyle{plain}

\setlength\marginparwidth{80pt}

\newcommand\marginlabel[1]{\mbox{}\marginpar{\raggedright\hspace{0pt}\textcolor{blue}{#1}}}


\newcommand{\reguaH}[1]{\noindent \rule{\linewidth}{#1mm}}

%%% CONTADORES

%%% Contador dos Algoritmos por cap�tulo

\newcounter{algo}[chapter]
\renewcommand\thealgo{\thechapter.\arabic{algo}}


\newenvironment{Algoritmo}[1] 
{\noindent \reguaH{0.1}  \colorbox{red}{\textbf{\textcolor[rgb]{1.00,1.00,1.00}
{Algoritmo \stepcounter{algo} \thealgo}}}\\ 
\par #1 \par  \hspace*{12cm} \colorbox{red}{\textbf{\textcolor[rgb]{1.00,1.00,1.00}{ \thealgo}}}\\ \reguaH{0.1} }
{}




%%% Contador dos Exemplos por cap�tulo

\newcounter{exemplo}[chapter]
\renewcommand\theexemplo{\thechapter.\arabic{exemplo}}


%\newenvironment{Exemplo}[1] 
%{\noindent \reguaH{0.1}  \colorbox{blue}{\textbf{\textcolor[rgb]{1.00,1.00,1.00}
%{Exemplo \stepcounter{exemplo} \theexemplo}}}\\ 
%\par #1 \par  \hspace*{12cm} \colorbox{blue}{\textbf{\textcolor[rgb]{1.00,1.00,1.00}{ \theexemplo}}}\\ \reguaH{0.1} }
%{}


\newenvironment{Exemplo}
{\noindent  \colorbox{blue}{\textbf{\textcolor[rgb]{1.00,1.00,1.00}
{Exemplo \stepcounter{exemplo} \theexemplo}}}\\ \reguaH{0.1} \\ \vspace*{0.5cm}}
{\hspace*{12cm} {\colorbox{blue}{\textbf{\textcolor[rgb]{1.00,1.00,1.00}{ \theexemplo}}}}\\ \noindent \reguaH{0.1} \par}







%%% Contador para Teoremas

\newcounter{teorema}[chapter]
\renewcommand\theteorema{\thechapter.\arabic{teorema}}

\newenvironment{Teorema}[1] 
{\noindent \reguaH{0.1}  \colorbox{black}{\textbf{\textcolor[rgb]{1.00,1.00,1.00}
{Teorema \stepcounter{teorema} \theteorema}}}\\ 
\par #1 \par  \hspace*{12cm} \colorbox{black}{\textbf{\textcolor[rgb]{1.00,1.00,1.00}{ \theteorema}}}\\ \reguaH{0.1} }
{}




%%% Contador para defini\c c�es

\newcounter{defin}[chapter]
\renewcommand\thedefin{\thechapter.\arabic{defin}}

\newenvironment{Defin}[1] 
{\noindent \reguaH{0.1}  \colorbox{yellow}{\textbf{\textcolor[rgb]{1.00,1.00,1.00}
{Defini\\c c�o\stepcounter{defin} \thedefin}}}\\ 
\par #1 \par  \hspace*{12cm} \colorbox{yellow}{\textbf{\textcolor[rgb]{1.00,1.00,1.00}{ \thedefin}}}\\ \reguaH{0.1} }
{}

%%% Contador dos Exerc�cios finais por cap�tulo

\newcounter{exercicio}[chapter]
\renewcommand\theexercicio{\thechapter.\arabic{exercicio}}

\newenvironment{Exercicio}
{\noindent \colorbox{blue}{\textbf{\textcolor[rgb]{1.00,1.00,1.00}
{\refstepcounter{exercicio} Exerc�cio \theexercicio}}}}
{ \vspace*{0.5cm} \par }


\newenvironment{sol}
{\noindent  \colorbox{magenta}{\textbf{\textcolor[rgb]{1.00,1.00,1.00}
{ Solu\c c�o}}}}
{ \vspace*{0.2cm} \par}

%% Contador para caixas

\newcounter{caixa}[chapter]
\renewcommand\thecaixa{\thechapter.\arabic{caixa}}

\newenvironment{Caixa}[1]
{\vspace*{0.5cm}
 \noindent \colorbox{blue}{\textbf{\textcolor[rgb]{1.00,1.00,1.00}
{Caixa\stepcounter{caixa} \thecaixa}}}\\ 
\noindent \doublebox{\parbox{\textwidth}{
#1} }   }
 {}




%----------------------------------------------

\definecolor{cinza}{rgb}{0.9,0.9,0.9}

\definecolor{cinza_claro}{rgb}{0.8,0.8,0.8}

\definecolor{orange}{rgb}{1,0.5,0.0}

\definecolor{rose}{rgb}{1,0.6,0.6}

\definecolor{shadecolor}{named}{cinza}

\definecolor{melao}{rgb}{1.0,0.8,0.4}

\newcommand{\problema}[1]{\noindent \fbox{} \colorbox{cinza}{\textbf{\textcolor[rgb]{1.00,1.00,1.00}{Problema #1}}}}


\newcommand{\modulo}[1]{\noindent\colorbox{cinza}{\textbf{\textcolor[rgb]{1.00,1.00,1.00}{M�dulo #1}}}}

\newcommand{\muitofacil}{\noindent \colorbox{magenta}{\textbf{\textcolor[rgb]{1.00,1.00,1.00}{MF}}}}


\newcommand{\facil}{\noindent \colorbox{green}{\textbf{\textcolor[rgb]{1.00,1.00,1.00}{F}}}}

\newcommand{\medio}{\noindent \colorbox{orange}{\textbf{\textcolor[rgb]{1.00,1.00,1.00}{M}}}}

\newcommand{\dificil}{\noindent \colorbox{red}{\textbf{\textcolor[rgb]{1.00,1.00,1.00}{D}}}}

\newcommand{\muitodificil}{\noindent \colorbox{rose}{\textbf{\textcolor[rgb]{1.00,1.00,1.00}{MD}}}}
%%% FIM DOS CONTADORES

% Para controlar o que aparece no �ndice 
\setcounter{tocdepth}{1}

%%% Para as notas 
%%% A definir


%\includeonly{cap06/cap06}

